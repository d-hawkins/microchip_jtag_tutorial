% =============================================================================
\section{JTAG-to-Register}
% =============================================================================
\label{sec:jtag_to_register}

% -----------------------------------------------------------------------------
\subsection{Architecture}
% -----------------------------------------------------------------------------

% -----------------------------------------------------------------------------
% JTAG-to-Register Block Diagram
% -----------------------------------------------------------------------------
%
\begin{figure}[t]
  \begin{center}
    \includegraphics[width=\textwidth]
    {figures/ujtag_to_register_diagram.pdf}
  \end{center}
  \caption{JTAG-to-Register block diagram.}
  \label{fig:jtag_to_register_diagram}
\end{figure}
% -----------------------------------------------------------------------------

The JTAG-to-Register component implements a single write (control) and
read (status) register. The register bit-width is controlled by a
SystemVerilog parameter, WIDTH. Figure~\ref{fig:jtag_to_register_diagram}
shows a block diagram of the architecture.
%
The JTAG-to-Register component uses a single-bit input select bit.
%
Figure~\ref{fig:jtag_to_register_diagram} shows the select bit will
assert for any valid user instruction (0x10 to 0x7F). User designs
could use the IR to select (and multiplex UTDO)  different blocks of
user logic.

Figure~\ref{fig:jtag_to_register_timing_ir} shows the instruction
shift-register timing. The TDO data shown in the figure is the
value read from the PolarFire SoC Discovery Kit.
%
Figure~\ref{fig:jtag_to_register_timing_ir} shows the data
shift-register timing. The figure shows TDI serializing 0x55 and
TDO serializing 0x11. These timing diagrams are reproduced by the
JTAG-to-Register testbench, and are discussed in the
Questasim waveform section.

% -----------------------------------------------------------------------------
% JTAG-to-Register Shift-IR
% -----------------------------------------------------------------------------
%
\begin{landscape}
\begin{figure}[p]
  \begin{center}
    \includegraphics[width=210mm]
    {figures/ujtag_to_register_timing_ir.pdf}
  \end{center}
  \caption{JTAG-to-Register JTAG instruction shift-register timing.}
  \label{fig:jtag_to_register_timing_ir}
\end{figure}
\end{landscape}
% -----------------------------------------------------------------------------

% -----------------------------------------------------------------------------
% JTAG-to-Register Shift-DR
% -----------------------------------------------------------------------------
%
\begin{landscape}
\begin{figure}[p]
  \begin{center}
    \includegraphics[width=210mm]
    {figures/ujtag_to_register_timing_dr.pdf}
  \end{center}
  \caption{JTAG-to-Register JTAG data shift-register timing.}
  \label{fig:jtag_to_register_timing_dr}
\end{figure}
\end{landscape}
% -----------------------------------------------------------------------------

\clearpage
% -----------------------------------------------------------------------------
\subsection{Testbench}
% -----------------------------------------------------------------------------

The JTAG-to-Register component is general-purpose, so is located in the
repository intellectual property (IP) directory \verb+ip/jtag_to_register+.
The JTAG-to-Register testbench is run as follows;
%
\begin{enumerate}
\item Start Questasim (2023.4 was used for this tutorial)
\item Change directory to the project, eg.,
\begin{verbatim}
vsim> cd {C:\github\microchip_jtag_tutorial\ip\jtag_to_register}
\end{verbatim}
\item Source the simulation script, eg.,
\begin{verbatim}
vsim> source scripts/questasim.tcl
\end{verbatim}
The script ends with a list of the testbench procedures
\begin{verbatim}
# Testbench Procedures
# --------------------
#
#  jtag_to_register_tb - run the JTAG-to-Register testbench
#
\end{verbatim}
\item Run the testbench
\begin{verbatim}
vsim> jtag_to_register_tb
\end{verbatim}
\end{enumerate}
%
The testbench uses the SmartFusion2 UJTAG component.

\clearpage
% -----------------------------------------------------------------------------
\subsection{Waveforms}
% -----------------------------------------------------------------------------

% -----------------------------------------------------------------------------
% JTAG-to-Register Questasim Waveforms
% -----------------------------------------------------------------------------
%
\begin{figure}[t]
  \begin{center}
    \includegraphics[width=\textwidth]
    {figures/jtag_to_register_questasim_waveforms.png}
  \end{center}
  \caption{JTAG-to-Register testbench Questasim waveforms.}
  \label{fig:jtag_to_register_questasim_waveforms}
\end{figure}
% -----------------------------------------------------------------------------

Figure~\ref{fig:jtag_to_register_questasim_waveforms} shows the Questasim
waveform view captured from the JTAG-to-Register testbench, with the
JTAG-to-Register component configured for 8-bits control/status width.
%
The testbench interfaces to the JTAG signals and implements the following
sequence;
%
\begin{itemize}
\item TAP reset
\item Load the 8-bit instruction register with 0x20
\item Load the 8-bit data register with 0x11 = 0001\_0001b
\item Load the 8-bit data register with 0x55 = 0101\_0101b
\item Load the 8-bit data register with 0x99 = 1001\_1001b
\item TAP reset
\end{itemize}
%
Figure~\ref{fig:jtag_to_register_questasim_waveforms} shows the data
register load with 0x55. Items of interest in the
waveform are;
%
\begin{itemize}
\item TDI contains the LSB-to-MSB serial data 0x55 = 0101\_0101b
\item TDO contains the LSB-to-MSB serial data 0x11 = 0001\_0001b
\item UTDO updates on the rising-edge of TCK
\item TDO updates on the falling-edge of TCK
\item status is captureed during the CAPTURE TAP state
\item control is updated during the UPDATE TAP state
\end{itemize}
%

\clearpage
% -----------------------------------------------------------------------------
\subsection{Software Interfacing with STAPL}
% -----------------------------------------------------------------------------

% -----------------------------------------------------------------------------
% Polarfire SoC Discovery and the Digilent Arty board
% -----------------------------------------------------------------------------
%
\begin{figure}[t]
  \begin{center}
    \includegraphics[width=\textwidth]
    {figures/pfs_disco_and_arty.png}\\
  \end{center}
  \caption{PolarFire SoC Discovery Kit and the Digilent Arty Board.}
  \label{fig:pfs_disco_and_arty}
\end{figure}
% -----------------------------------------------------------------------------

STAPL (Standard Test and Programming Language) is an Altera-developed standard
for JTAG interfacing, standardized in JEDEC standard
JESD-71~\cite{JEDEC_JESD71_1999}.
%
Two STAPL files are located in the repository:
%
\begin{itemize}
\item \verb+stp/read_idcode.stp+

Read and print a Microchip JTAG device IDCODE.

This script resets the TAP, loads the instruction register with the 8-bit
read IDCODE instruction, and then reads a single 32-bit IDCODE.

\item \verb+stp/jtag_to_register.stp+

Incrementing LED count.

This script resets the TAP, loads the instruction register with the 8-bit
user instruction 0x20, and then loops from 1 to 15, loading the data register
with the loop index, and printing the value read (which matches the previously
loaded loop index).

\end{itemize}
%
The first version of the JTAG-to-Register STAPL was based on the example in
AC227 Appendix A~\cite{Microchip_AC227_2015}. The JEDEC STAPL specification
was then consulted to understand the syntax for: loops, integer-to-binary
conversion, binary-to-integer conversion, and printing messages.

Figure~\ref{fig:pfs_disco_and_arty} shows the hardware setup used to
capture logic analyzer traces. The PolarFire SoC Discovery Kit Raspberry
Pi connector drove UJTAG signals to a PMod connector on the
Digilent Arty board, and the Xilinx Vivado
\href{https://www.xilinx.com/products/intellectual-property/ila.html}
{Integrated Logic Analyzer (ILA)} was used to capture traces.
%
FlashPro Express was used to execute the JTAG-to-Register STAPL file.
Figure~\ref{fig:jtag_to_register_stapl} shows logic analyzer traces for
TDI serializing 0x02, 0x03, and 0x04, and TDO serializing 0x01, 0x02, and
0x03.

\clearpage
% -----------------------------------------------------------------------------
\subsection{Software Interfacing with FTDI D2XX}
% -----------------------------------------------------------------------------

A JTAG-to-Register application was written in C++, compiled with Microsoft
Visual Studio 2022 (Community Edition), and linked with the FTDI D2XX DLL
(the application is part of the FTDI D2XX tutorial).
%
The application used the FTDI Multi-Protocol Synchronous Serial Engine
(MPSSE) mode to implement JTAG transactions.
%
Figure~\ref{fig:jtag_to_register_d2xx} shows logic analyzer traces for
TDI serializing 0x02, 0x03, and 0x04, and TDO serializing 0x01, 0x02, and
0x03. The logic analyzer traces are very similar to those captured from the
FlashPro Express STAPL transactions in
Figure~\ref{fig:jtag_to_register_stapl}, with one minor difference in
that the STAPL captures show additional TCK pulses while in Run-Test-Idle.

% -----------------------------------------------------------------------------
% JTAG-to-Register on the Polarfire SoC
% -----------------------------------------------------------------------------
%
\begin{figure}[p]
  \begin{center}
    \includegraphics[width=0.95\textwidth]
    {figures/jtag_to_register_stapl_0x02_0x01.png}\\
    (a) Write 0x02 = 0000\_0010b. Read 0x01 = 0000\_0001b
  \end{center}
  \vskip5mm
  \begin{center}
    \includegraphics[width=0.95\textwidth]
    {figures/jtag_to_register_stapl_0x03_0x02.png}\\
    (b) Write 0x03 = 0000\_0011b. Read 0x02 = 0000\_0010b
  \end{center}
  \vskip5mm
  \begin{center}
    \includegraphics[width=0.95\textwidth]
    {figures/jtag_to_register_stapl_0x04_0x03.png}\\
    (c) Write 0x04 = 0000\_0100b. Read 0x03 = 0000\_0011b
  \end{center}
  \caption{JTAG-to-Register logic analyzer traces for STAPL interfacing.}
  \label{fig:jtag_to_register_stapl}
\end{figure}
% -----------------------------------------------------------------------------

% -----------------------------------------------------------------------------
% JTAG-to-Register on the Polarfire SoC
% -----------------------------------------------------------------------------
%
\begin{figure}[p]
  \begin{center}
    \includegraphics[width=0.95\textwidth]
    {figures/jtag_to_register_d2xx_0x02_0x01.png}\\
    (a) Write 0x02 = 0000\_0010b. Read 0x01 = 0000\_0001b
  \end{center}
  \vskip5mm
  \begin{center}
    \includegraphics[width=0.95\textwidth]
    {figures/jtag_to_register_d2xx_0x03_0x02.png}\\
    (b) Write 0x03 = 0000\_0011b. Read 0x02 = 0000\_0010b
  \end{center}
  \vskip5mm
  \begin{center}
    \includegraphics[width=0.95\textwidth]
    {figures/jtag_to_register_d2xx_0x04_0x03.png}\\
    (c) Write 0x04 = 0000\_0100b. Read 0x03 = 0000\_0011b
  \end{center}
  \caption{JTAG-to-Register logic analyzer traces for FTDI D2XX interfacing.}
  \label{fig:jtag_to_register_d2xx}
\end{figure}
% -----------------------------------------------------------------------------



% -----------------------------------------------------------------------------
\subsection{Hardware Implementations}
% -----------------------------------------------------------------------------

Table~\ref{tab:jtag_to_register_hardware} shows the JTAG-to-Register
design was implemented on the following hardware:
%
\begin{itemize}
\item Microchip Polarfire SoC Discovery Kit
(\href{https://www.microchip.com/en-us/development-tool/mpfs-disco-kit}
{MPFS-DISCO-KIT})

\item Microchip ProASIC3 Starter Kit
(\href{https://www.microchip.com/en-us/development-tool/a3pe-starter-kit-2}
{A3PE-STARTER-KIT-2})

\item Microchip SmartFusion2 Security Evaluation Kit
(\href{https://www.microchip.com/en-us/development-tool/m2s090ts-eval-kit}
{M2S090TS-EVAL-KIT})

\item Emcraft SmartFusion2 Starter Kit
(\href{https://emcraft.com/products/153}{SF2-STARTER-KIT})

The tutorial used the ES version of this kit: SF2-STARTER-KIT-ES

\item Avnet SmartFusion2 Kickstart Kit (no longer available)

\end{itemize}
%
The UJTAG component is supported by other Microchip FPGAs, eg.,
\href{https://www.microchip.com/en-us/development-tool/m2gl-eval-kit}{IGLOO2}.

% -----------------------------------------------------------------------------
% Hardware implementations
% -----------------------------------------------------------------------------
%
\begin{table}
\caption{JTAG-to-Register hardware implementations.}
\label{tab:jtag_to_register_hardware}
\begin{center}
\begin{tabular}{|l|l|l|}
\hline
Directory & Board & Control\\
\hline\hline
&&\\
\texttt{designs/pfs\_disco}     & PolarFire SoC Discovery & 4 LEDs\\
\texttt{designs/pa3\_starter}   & ProASIC3 Starter        & 7 LEDs\\
\texttt{designs/sf2\_security}  & SmartFusion2 Security   & 7 LEDs\\
\texttt{designs/sf2\_starter}   & SmartFusion2 Starter    & 1 LED\\
\texttt{designs/sf2\_kickstart} & SmartFusion2 Kickstart  & 3 LEDs\\
&&\\
\hline
\end{tabular}
\end{center}
\end{table}
% -----------------------------------------------------------------------------
