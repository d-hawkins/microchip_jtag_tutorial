% =============================================================================
\section{UJTAG}
% =============================================================================
\label{sec:ujtag}

% -----------------------------------------------------------------------------
\subsection{Overview}
% -----------------------------------------------------------------------------

The Microchip UJTAG component is described in
AC227~\cite{Microchip_AC227_2015} and in the
\href{https://www.microchip.com/en-us/products/fpgas-and-plds/fpgas/proasic-3-fpgas}
{ProASIC3} Fabric Users Guide~\cite{Microchip_PA3_UG_2012}.
%
The UJTAG component is also described in the Macro Library Users Guide for
each device.
%
Figure~\ref{fig:polarfire_macro_ujtag} shows the Microchip UJTAG component from
the PolarFire Macro Library in the Libero SoC online documentation for v2024.2.
%
UJTAG is also supported in
\href{https://www.microchip.com/en-us/products/fpgas-and-plds/fpgas/igloo-2-fpgas}
{IGLOO2},
\href{https://www.microchip.com/en-us/products/fpgas-and-plds/system-on-chip-fpgas/smartfusion-2-fpgas}
{SmartFusion2},
\href{https://www.microchip.com/en-us/products/fpgas-and-plds/fpgas/polarfire-fpgas/polarfire-mid-range-fpgas}
{Polarfire}, and
\href{https://www.microchip.com/en-us/products/fpgas-and-plds/system-on-chip-fpgas/polarfire-soc-fpgas}
{PolarFire SoC} devices.
%
The example designs in this tutorial target evaluation boards for
the ProASIC3, SmartFusion2, and PolarFire SoC devices.

The UJTAG component provides user logic access to the JTAG clock, serial data
in and out, the 8-bit instruction register, and four TAP state indicators.
The serial data in and TAP state indicators change on the falling-edge of
the JTAG clock.
%
A user interface to the UJTAG component typically consists of;
%
\begin{itemize}
\item A shift-register used for serial data in and out (shifted LSB-to-MSB)
\item URSTB used as asynchronous reset
\item UDRCK used as the clock
\item UIREG[7:0] used as a select (or multiplex) control
\item UDRCAP used to load the shift-register
\item UDRSH used to enable the shift-register
\item UDRUPD used to capture the shift-register into a parallel register
\item UTDO driven by the shift-register LSB
\end{itemize}
%
A user interface to the UJTAG component does not have to use all UJTAG signals.
For example, an SPI-like interface can be constructed using;
%
\begin{itemize}
\item SPI clock is UDRCK
\item SPI select generated from a UIREG pattern (or bit) and UDRSH
\item SPI MOSI is UTDI
\item SPI MISO is UTDO
\end{itemize}
%
The SPI analogy works well for hardware designs containing only a single
device in the JTAG chain. The serial data stream is slightly different
for hardware designs containing multiple devices daisy-chained in the JTAG chain.
When communicating with a single device on a multi-device JTAG chain, the other
devices are placed in BYPASS mode, and each bypassed device introduces a
zero valued bit into the serial data stream. These extra bits can be handled
by detecting the number of devices in BYPASS mode before, and after, the
selected device.

The UJTAG UTDO input must be driven on the rising-edge of UDRCK, as the UJTAG
contains a falling-edge register on the path to the JTAG TDO output.
Figure~\ref{fig:jtag_to_register_questasim_waveforms} shows the Questasim
simulation waveforms of the JTAG-to-Register design. The serial data on TDI
is 0x55 = 0101\_0101b, while the serial data out on TDO is 0x11 = 0001\_0001b.
The UTDO data, highlighted in cyan, is clocked on the rising-edge of DRCK
within the fabric logic. The TDO changes are observed to occur on the
falling-edge of TCK.
%
The operation of UTDO is inconsistent with AC227, where Figure 4 on p3 indicates
that UTDO = TDO~\cite{Microchip_AC227_2015}, and is inconsistent with the ProASIC3
fabric users guide statement on p381 that \emph{UTDI, UTDO, and UDRCK are directly
connected to the JTAG TDI, TDO, and TCK ports, respectively.}~\cite{Microchip_PA3_UG_2012}.
%
The UJTAG simulation models for the ProASIC3 (Libero IDE 9.2 SP4 \verb+proasic3e.v+ library)
and for the SmartFusion2 (Libero SoC 2024.1 \verb+smartfusion2.v+ library) both
exhibit the same UTDO-to-TDO pipelining through a falling-edge register
(see the testbenches in \verb+ip/ujtag+).

Figures~\ref{fig:sf2_starter_ujtag_ila_tdo_rise}
and~\ref{fig:sf2_starter_ujtag_ila_tdo_fall} show logic analyzer traces
captured from the SmartFusion2 Starter Kit for UJTAG designs that
drove UTDO on the rising-edge and falling-edge of UDRCK respectively.
The SmartFusion 2 Starter Kit was used, as it contains both an FPGA
JTAG header and an ARM debug header, so the JTAG signals could be
probed on the ARM header. The JTAG communications were implemented by
an FTDI D2XX applications. The logic analyzer traces were captured using
the Digilent Arty board and Xilinx Vivado.
%
\begin{itemize}
\item \textcolor{OliveGreen}{\textbf{TDO driven on UDRCK rising-edge}}

The TDO bits in Figure~\ref{fig:sf2_starter_ujtag_ila_tdo_rise}
match the expected values.

\item \textcolor{red}{\textbf{TDO driven on UDRCK falling-edge}}

The TDO bits in Figure~\ref{fig:sf2_starter_ujtag_ila_tdo_fall} do not match
the expected values: the expected bit values are delayed by 1-bit,
with the 2 LSBs repeated.

\end{itemize}
%
\vskip5mm
\begin{center}
\textcolor{magenta}{\textbf{Conclusion:} UTDO must be clocked on UDRCK rising-edge.}
\end{center}

% -----------------------------------------------------------------------------
% PolarFire Macro Library UJTAG
% -----------------------------------------------------------------------------
%
\begin{figure}[t]
  \begin{minipage}{0.5\textwidth}
  \begin{center}
    \includegraphics[width=0.7\textwidth]
    {figures/polarfire_macro_ujtag.pdf}\\
    (a) Component
  \end{center}
  \end{minipage}
  \hfil
  \begin{minipage}{0.5\textwidth}
  \begin{center}
  \begin{tabular}{l|l}
  Pin & Function\\
  \hline
  &\\
  &\textbf{JTAG}\\
  TRSTB      & Reset (active low)\\
  TCK        & Clock\\
  TMS        & Test mode select\\
  TDI        & Serial data in\\
  TDO        & Serial data out\\
  &\\
  &\textbf{TAP State}\\
  URSTB      & Test-Logic-Reset\\
  UDRCAP     & Capture-DR\\
  UDRSH      & Shift-DR\\
  UDRUPD     & Update-DR\\
  &\\
  &\textbf{User}\\
  UIREG[7:0] & Instruction\\
  UDRCK      & Clock\\
  UTDI       & Serial data in\\
  UTDO       & Serial data out\\
  &\\
  \hline
  \end{tabular}\\
  \vskip1mm
  (b) Pin functions
  \end{center}
  \end{minipage}
  \caption{Microchip UJTAG component.}
  \label{fig:polarfire_macro_ujtag}
\end{figure}
% -----------------------------------------------------------------------------

\clearpage
% -----------------------------------------------------------------------------
% SmartFusion2 Starter Kit ILA traces for UTDO updated on DRCK rising-edge
% -----------------------------------------------------------------------------
%
\begin{figure}[p]
  \begin{center}
    \includegraphics[width=0.95\textwidth]
    {figures/sf2_starter_ujtag_ila_tdo_rise_0x09_0x08.png}\\
    (a) Write 0x09 = 0000\_\textcolor{blue}{1001}b. Read 0x08 = 0000\_\textcolor{blue}{1000}b
  \end{center}
%  \vskip5mm
  \begin{center}
    \includegraphics[width=0.95\textwidth]
    {figures/sf2_starter_ujtag_ila_tdo_rise_0x0A_0x09.png}\\
    (b) Write 0x0A = 0000\_\textcolor{blue}{1010}b. Read 0x09 = 0000\_\textcolor{blue}{1001}b
  \end{center}
%  \vskip5mm
  \begin{center}
    \includegraphics[width=0.95\textwidth]
    {figures/sf2_starter_ujtag_ila_tdo_rise_0x0B_0x0A.png}\\
    (c) Write 0x0B = 0000\_\textcolor{blue}{1011}b. Read 0x0A = 0000\_\textcolor{blue}{1010}b
  \end{center}
%  \vskip5mm
  \begin{center}
    \includegraphics[width=0.95\textwidth]
    {figures/sf2_starter_ujtag_ila_tdo_rise_0x0C_0x0B.png}\\
    (d) Write 0x0C = 0000\_\textcolor{blue}{1100}b. Read 0x0B = 0000\_\textcolor{blue}{1011}b
  \end{center}
  \caption{JTAG logic analyzer traces for UTDO driven on UDRCK rising-edge.}
  \label{fig:sf2_starter_ujtag_ila_tdo_rise}
\end{figure}
% -----------------------------------------------------------------------------

\clearpage
% -----------------------------------------------------------------------------
% SmartFusion2 Starter Kit ILA traces for UTDO updated on DRCK falling-edge
% -----------------------------------------------------------------------------
%
\begin{figure}[p]
  \begin{center}
    \includegraphics[width=0.95\textwidth]
    {figures/sf2_starter_ujtag_ila_tdo_fall_0x09_0x10.png}\\
    (a) Write 0x09 = 0000\_\textcolor{blue}{1001}b. Read 0x10 = 000\textcolor{blue}{1\_000}\textcolor{red}{0}b
  \end{center}
%  \vskip5mm
  \begin{center}
    \includegraphics[width=0.95\textwidth]
    {figures/sf2_starter_ujtag_ila_tdo_fall_0x0A_0x13.png}\\
    (b) Write 0x0A = 0000\_\textcolor{blue}{1010}b. Read 0x13 = 000\textcolor{blue}{1\_001}\textcolor{red}{1}b
  \end{center}
%  \vskip5mm
  \begin{center}
    \includegraphics[width=0.95\textwidth]
    {figures/sf2_starter_ujtag_ila_tdo_fall_0x0B_0x14.png}\\
    (c) Write 0x0B = 0000\_\textcolor{blue}{1011}b. Read 0x14 = 000\textcolor{blue}{1\_010}\textcolor{red}{0}b
  \end{center}
%  \vskip5mm
  \begin{center}
    \includegraphics[width=0.95\textwidth]
    {figures/sf2_starter_ujtag_ila_tdo_fall_0x0C_0x17.png}\\
    (d) Write 0x0C = 0000\_\textcolor{blue}{1100}b. Read 0x17 = 000\textcolor{blue}{1\_011}\textcolor{red}{1}b
  \end{center}
  \caption{JTAG logic analyzer traces for UTDO driven on UDRCK falling-edge.}
  \label{fig:sf2_starter_ujtag_ila_tdo_fall}
\end{figure}
% -----------------------------------------------------------------------------

\clearpage
% -----------------------------------------------------------------------------
\subsection{UJTAG Testbench}
% -----------------------------------------------------------------------------

The simplest usage of the UJTAG component is a serial-to-parallel register
with write (host-to-FPGA) and read (FPGA-to-host) support. This sort of
interface is commonly found in SPI devices. For example, the Texas Instruments
LMX2615-SP clock synthesizer uses a 24-bit SPI transaction consisting of;
%
\begin{itemize}
\item 1-bit read(1)/write(0) indicator
\item 7-bit address (128 possible registers)
\item 16-bit data
\end{itemize}
%
Another example is the Texas Instruments LMK04832 clock synthesizer which
also uses a 24-bit SPI transaction, but with a slightly different bit
functionality;
%
\begin{itemize}
\item 1-bit read(1)/write(0) indicator
\item 15-bit address (32768 possible registers)
\item 8-bit data
\end{itemize}
%
The UJTAG component could be used to create this interface, but for the
purpose of this tutorial, just the serial-to-parallel register is
implemented, without interpretation of the read/write register bits.

The UJTAG testbench is located in the repository directory \verb+ip/ujtag+.
The UJTAG testbench is run as follows;
%
\begin{enumerate}
\item Start Questasim (2023.4 was used for this tutorial)
\item Change directory to the project, eg.,
\begin{verbatim}
vsim> cd {C:\github\microchip_jtag_tutorial\ip\ujtag}
\end{verbatim}
\item Source the simulation script, eg.,
\begin{verbatim}
vsim> source scripts/questasim.tcl
\end{verbatim}
The script ends with a list of the testbench procedures
\begin{verbatim}
# Testbench Procedures
# --------------------
#
#  ujtag_pa3_tb - run the UJTAG ProASIC3 testbench
#  ujtag_sf2_tb - run the UJTAG Smartfusion2 (and newer) testbench
#
\end{verbatim}
\item Run the ProASIC3 UJTAG testbench
\begin{verbatim}
vsim> ujtag_pa3_tb
\end{verbatim}
\item Run the SmartFusion2 UJTAG testbench
\begin{verbatim}
vsim> ujtag_sf2_tb
\end{verbatim}
\end{enumerate}
%
There are two UJTAG testbenches as the ProASIC3 (and earlier) devices used
a slightly different port naming convention than the SmartFusion2 (and newer)
devices. A UJTAG wrapper component is used to update the ProASIC3 port mapping
to match that of the SmartFusion2 port mapping, and the testbench instantiates
the wrapper. See the source code for additional details.

The UJTAG testbench implements the JTAG-to-Register component within the
testbench. The next section describes the JTAG-to-Register implemented as
a separate SystemVerilog module.
