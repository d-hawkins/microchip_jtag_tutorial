% -----------------------------------------------------------------------------
% microchip_jtag_tutorial.txt
%
% 1/15/2025 D. W. Hawkins (dwh@caltech.edu)
%
% Microchip JTAG tutorial.
%
% -----------------------------------------------------------------------------
%
% -----------------------------------------------------------------------------
% Document preamble
% -----------------------------------------------------------------------------
%
\documentclass[10pt,twoside]{article}

% Math symbols
\usepackage{amsmath}
\usepackage{amssymb}

% Headers/Footers
\usepackage{fancyhdr}

% Colors
% See https://en.wikibooks.org/wiki/LaTeX/Colors
\usepackage[dvipsnames]{xcolor}

% Importing and manipulating graphics
\usepackage{graphicx}
\usepackage{subfig}
\usepackage{pdflscape}

% Misc packages
\usepackage{verbatim}
\usepackage{dcolumn}
\usepackage{ifpdf}
\usepackage{enumerate}

% PDF Bookmarks and hyperref stuff
\usepackage[
  bookmarks=true,
  bookmarksnumbered=true,
  colorlinks=true,
  filecolor=blue,
  linkcolor=blue,
  urlcolor=blue,
  hyperfootnotes=true
  citecolor=blue
]{hyperref}

% Improved citation handling
% (include after the hyperref stuff)
\usepackage{cite}

% Pretty printing code
\usepackage{listings}

% Syntax highlighting
% * Use OliveGreen for the comments to make them darker green
\lstset{
	language=C,
	basicstyle=\small\ttfamily,
	keywordstyle=\color{blue}\ttfamily,
	stringstyle=\color{red}\ttfamily,
	commentstyle=\color{OliveGreen}\ttfamily,
	tabsize=4,
	showstringspaces=false
}

% -----------------------------------------------------------------------------
% Setup the margins
% -----------------------------------------------------------------------------
% Footer Template

% Set left margin - The default is 1 inch, so the following
% command sets a 1.25-inch left margin.
\setlength{\oddsidemargin}{0.25in}
\setlength{\evensidemargin}{0.25in}

% Set width of the text - What is left will be the right
% margin. In this case, right margin is
% 8.5in - 1.25in - 6in = 1.25in.
\setlength{\textwidth}{6in}

% Set top margin - The default is 1 inch, so the following
% command sets a 0.75-inch top margin.
%\setlength{\topmargin}{-0.25in}
\setlength{\topmargin}{-0.2in}

% Set height of the header
\setlength{\headheight}{0.3in}

% Set vertical distance between the header and the text
\setlength{\headsep}{0.2in}

% Set height of the text
\setlength{\textheight}{8.5in}

% Set vertical distance between the text and the
% bottom of footer
\setlength{\footskip}{0.4in}

% -----------------------------------------------------------------------------
% Allow floats to take up more space on a page.
% -----------------------------------------------------------------------------

% see page 142 of the Companion for this stuff and the
% documentation for the fancyhdr package
\renewcommand{\textfraction}{0.05}
\renewcommand{\topfraction}{0.95}
\renewcommand{\bottomfraction}{0.95}
% dont make this too small
\renewcommand{\floatpagefraction}{0.35}
\setcounter{totalnumber}{5}

% Make sure the top/bottom rules appear on the page
\renewcommand{\headrulewidth}{0.4pt}
\renewcommand{\footrulewidth}{0.4pt}

% -----------------------------------------------------------------------------
% Set up the header/footer
% -----------------------------------------------------------------------------
%
% First page
% - set the head/foot rule width to zero to hide them
\fancypagestyle{plain}{
	\renewcommand{\headrulewidth}{0pt}
	\renewcommand{\footrulewidth}{0pt}
	\fancyhf{}
	\fancyfoot{}
}

% All other pages
\lhead{Microchip JTAG}
\chead{}
\rhead{\today}
\lfoot{}
\cfoot{}
\rfoot{\thepage}

% =============================================================================
% Document contents
% =============================================================================
%
\begin{document}
\title{Microchip FPGA JTAG Tutorial}
\author{D. W. Hawkins (dwh@caltech.edu)}
\date{\today}

% Title page
\maketitle

% Table of contents
\tableofcontents

% Switch to the fancy page style
\pagestyle{fancy}

% Start the first section on an odd page
%\cleardoublepage
\clearpage

% =============================================================================
% Main Document
% =============================================================================
%
% =============================================================================
\section{Introduction}
% =============================================================================
\label{sec:intro}

This tutorial demonstrates how to use the Microchip FPGA UJTAG
component~\cite{Microchip_AC227_2015}.
%
The Microchip UJTAG component provides a communications path over the
JTAG\footnote{Joint Test Action Group.} interface to a user design
implemented in the programmable logic fabric of an
FPGA\footnote{Field Programmable Gate Array.}.
%
Similar JTAG-to-fabric interfaces from other FPGA vendors are;
Altera Virtual JTAG~\cite{Altera_VJTAG_2021} and
Xilinx BSCAN~\cite{Xilinx_UG908_2024}.
%
Figure~\ref{fig:jtag_tap} shows the JTAG (IEEE 1149.1) Test Access Port
(TAP) Controller. JTAG TAP state transitions are controlled by the JTAG
clock (TCK) and test mode select (TMS) control.

This tutorial includes:
%
\begin{itemize}
\item Hardware implementation of the UJTAG component
\item UJTAG software interfacing using FlashPro Express and STAPL
\item UJTAG software interfacing using FTDI D2XX under Windows and Linux
\end{itemize}
%
The FTDI D2XX~\cite{FTDI_D2XX_PG_2023} example applications were tested
under Windows 10 and Ubuntu 24.04\footnote{Ubuntu was tested using an
Oracle VirtualBox Virtual Machine (VM) running under Windows 10.}.

% -----------------------------------------------------------------------------
% JTAG TAP
% -----------------------------------------------------------------------------
\begin{figure}[t]
  \begin{center}
    \includegraphics[width=\textwidth]
    {figures/jtag_tap.pdf}
  \end{center}
  \caption{JTAG (IEEE 1149.1) Test Access Port (TAP) Controller~\cite{IEEE_STD_1149_1_2013}.}
  \label{fig:jtag_tap}
\end{figure}
% -----------------------------------------------------------------------------

\clearpage
% =============================================================================
\section{Resources}
% =============================================================================
\label{sec:resources}

Resources related to this tutorial are:
%
\begin{itemize}
\item Libero SoC Download:
%
\begin{itemize}
\item
\href{https://www.microchip.com/en-us/products/fpgas-and-plds/fpga-and-soc-design-tools/fpga/libero-software-later-versions}
{https://www.microchip.com/en-us/products/fpgas-and-plds/\newline
fpga-and-soc-design-tools/fpga/libero-software-later-versions}

\item
Libero SoC 2024.1 was used during the development of this tutorial.
\end{itemize}

\item Polarfire SoC Discovery Kit (MPFS-DISCO-KIT):
%
\begin{itemize}
\item
\href{https://www.microchip.com/en-us/development-tool/mpfs-disco-kit}
{https://www.microchip.com/en-us/development-tool/mpfs-disco-kit}
\end{itemize}

\item FTDI D2XX software tutorial
%
\begin{itemize}
\item
\href{https://github.com/d-hawkins/ftdi_d2xx_tutorial}
{https://github.com/d-hawkins/ftdi\_d2xx\_tutorial}
\end{itemize}

\item Altera JTAG tutorials
%
\begin{itemize}
\item
\href{https://github.com/d-hawkins/altera_jtag_to_avalon_mm_tutorial/blob/main/doc/altera_jtag_to_avalon_mm_tutorial.pdf}
{altera\_jtag\_to\_avalon\_mm\_tutorial.pdf}

\item
\href{https://github.com/d-hawkins/altera_jtag_to_avalon_analysis/blob/main/doc/altera_jtag_to_avalon_analysis.pdf}
{altera\_jtag\_to\_avalon\_analysis.pdf}

\item
\href{https://github.com/d-hawkins/altera_virtual_jtag_analysis/blob/main/doc/altera_virtual_jtag_analysis.pdf}
{altera\_virtual\_jtag\_analysis.pdf}
\end{itemize}

\item Xilinx JTAG tutorial (in progress)

\end{itemize}
%
The bibliography contains additional references.
\clearpage
% =============================================================================
\section{UJTAG}
% =============================================================================
\label{sec:ujtag}

% -----------------------------------------------------------------------------
\subsection{Overview}
% -----------------------------------------------------------------------------

The Microchip UJTAG component is described in
AC227~\cite{Microchip_AC227_2015} and in the
\href{https://www.microchip.com/en-us/products/fpgas-and-plds/fpgas/proasic-3-fpgas}
{ProASIC3} Fabric Users Guide~\cite{Microchip_PA3_UG_2012}.
%
The UJTAG component is also described in the Macro Library Users Guide for
each device.
%
Figure~\ref{fig:polarfire_macro_ujtag} shows the Microchip UJTAG component from
the PolarFire Macro Library in the Libero SoC online documentation for v2024.2.
%
UJTAG is also supported in
\href{https://www.microchip.com/en-us/products/fpgas-and-plds/fpgas/igloo-2-fpgas}
{IGLOO2},
\href{https://www.microchip.com/en-us/products/fpgas-and-plds/system-on-chip-fpgas/smartfusion-2-fpgas}
{SmartFusion2},
\href{https://www.microchip.com/en-us/products/fpgas-and-plds/fpgas/polarfire-fpgas/polarfire-mid-range-fpgas}
{Polarfire}, and
\href{https://www.microchip.com/en-us/products/fpgas-and-plds/system-on-chip-fpgas/polarfire-soc-fpgas}
{PolarFire SoC} devices.
%
The example designs in this tutorial target evaluation boards for
the ProASIC3, SmartFusion2, and PolarFire SoC devices.

The UJTAG component provides user logic access to the JTAG clock, serial data
in and out, the 8-bit instruction register, and four TAP state indicators.
The serial data in and TAP state indicators change on the falling-edge of
the JTAG clock.
%
A user interface to the UJTAG component typically consists of;
%
\begin{itemize}
\item A shift-register used for serial data in and out (shifted LSB-to-MSB)
\item URSTB used as asynchronous reset
\item UDRCK used as the clock
\item UIREG[7:0] used as a select (or multiplex) control
\item UDRCAP used to load the shift-register
\item UDRSH used to enable the shift-register
\item UDRUPD used to capture the shift-register into a parallel register
\item UTDO driven by the shift-register LSB
\end{itemize}
%
A user interface to the UJTAG component does not have to use all UJTAG signals.
For example, an SPI-like interface can be constructed using;
%
\begin{itemize}
\item SPI clock is UDRCK
\item SPI select generated from a UIREG pattern (or bit) and UDRSH
\item SPI MOSI is UTDI
\item SPI MISO is UTDO
\end{itemize}
%
The SPI analogy works well for hardware designs containing only a single
device in the JTAG chain. The serial data stream is slightly different
for hardware designs containing multiple devices daisy-chained in the JTAG chain.
When communicating with a single device on a multi-device JTAG chain, the other
devices are placed in BYPASS mode, and each bypassed device introduces a
zero valued bit into the serial data stream. These extra bits can be handled
by detecting the number of devices in BYPASS mode before, and after, the
selected device.

The UJTAG UTDO input must be driven on the rising-edge of UDRCK, as the UJTAG
contains a falling-edge register on the path to the JTAG TDO output.
Figure~\ref{fig:jtag_to_register_questasim_waveforms} shows the Questasim
simulation waveforms of the JTAG-to-Register design. The serial data on TDI
is 0x55 = 0101\_0101b, while the serial data out on TDO is 0x11 = 0001\_0001b.
The UTDO data, highlighted in cyan, is clocked on the rising-edge of DRCK
within the fabric logic. The TDO changes are observed to occur on the
falling-edge of TCK.
%
The operation of UTDO is inconsistent with AC227, where Figure 4 on p3 indicates
that UTDO = TDO~\cite{Microchip_AC227_2015}, and is inconsistent with the ProASIC3
fabric users guide statement on p381 that \emph{UTDI, UTDO, and UDRCK are directly
connected to the JTAG TDI, TDO, and TCK ports, respectively.}~\cite{Microchip_PA3_UG_2012}.
%
The UJTAG simulation models for the ProASIC3 (Libero IDE 9.2 SP4 \verb+proasic3e.v+ library)
and for the SmartFusion2 (Libero SoC 2024.1 \verb+smartfusion2.v+ library) both
exhibit the same UTDO-to-TDO pipelining through a falling-edge register
(see the testbenches in \verb+ip/ujtag+).

Figures~\ref{fig:sf2_starter_ujtag_ila_tdo_rise}
and~\ref{fig:sf2_starter_ujtag_ila_tdo_fall} show logic analyzer traces
captured from the SmartFusion2 Starter Kit for UJTAG designs that
drove UTDO on the rising-edge and falling-edge of UDRCK respectively.
The SmartFusion 2 Starter Kit was used, as it contains both an FPGA
JTAG header and an ARM debug header, so the JTAG signals could be
probed on the ARM header. The JTAG communications were implemented by
an FTDI D2XX applications. The logic analyzer traces were captured using
the Digilent Arty board and Xilinx Vivado.
%
\begin{itemize}
\item \textcolor{OliveGreen}{\textbf{TDO driven on UDRCK rising-edge}}

The TDO bits in Figure~\ref{fig:sf2_starter_ujtag_ila_tdo_rise}
match the expected values.

\item \textcolor{red}{\textbf{TDO driven on UDRCK falling-edge}}

The TDO bits in Figure~\ref{fig:sf2_starter_ujtag_ila_tdo_fall} do not match
the expected values: the expected bit values are delayed by 1-bit,
with the 2 LSBs repeated.

\end{itemize}
%
\vskip5mm
\begin{center}
\textcolor{magenta}{\textbf{Conclusion:} UTDO must be clocked on UDRCK rising-edge.}
\end{center}

% -----------------------------------------------------------------------------
% PolarFire Macro Library UJTAG
% -----------------------------------------------------------------------------
%
\begin{figure}[t]
  \begin{minipage}{0.5\textwidth}
  \begin{center}
    \includegraphics[width=0.7\textwidth]
    {figures/polarfire_macro_ujtag.pdf}\\
    (a) Component
  \end{center}
  \end{minipage}
  \hfil
  \begin{minipage}{0.5\textwidth}
  \begin{center}
  \begin{tabular}{l|l}
  Pin & Function\\
  \hline
  &\\
  &\textbf{JTAG}\\
  TRSTB      & Reset (active low)\\
  TCK        & Clock\\
  TMS        & Test mode select\\
  TDI        & Serial data in\\
  TDO        & Serial data out\\
  &\\
  &\textbf{TAP State}\\
  URSTB      & Test-Logic-Reset\\
  UDRCAP     & Capture-DR\\
  UDRSH      & Shift-DR\\
  UDRUPD     & Update-DR\\
  &\\
  &\textbf{User}\\
  UIREG[7:0] & Instruction\\
  UDRCK      & Clock\\
  UTDI       & Serial data in\\
  UTDO       & Serial data out\\
  &\\
  \hline
  \end{tabular}\\
  \vskip1mm
  (b) Pin functions
  \end{center}
  \end{minipage}
  \caption{Microchip UJTAG component.}
  \label{fig:polarfire_macro_ujtag}
\end{figure}
% -----------------------------------------------------------------------------

\clearpage
% -----------------------------------------------------------------------------
% SmartFusion2 Starter Kit ILA traces for UTDO updated on DRCK rising-edge
% -----------------------------------------------------------------------------
%
\begin{figure}[p]
  \begin{center}
    \includegraphics[width=0.95\textwidth]
    {figures/sf2_starter_ujtag_ila_tdo_rise_0x09_0x08.png}\\
    (a) Write 0x09 = 0000\_\textcolor{blue}{1001}b. Read 0x08 = 0000\_\textcolor{blue}{1000}b
  \end{center}
%  \vskip5mm
  \begin{center}
    \includegraphics[width=0.95\textwidth]
    {figures/sf2_starter_ujtag_ila_tdo_rise_0x0A_0x09.png}\\
    (b) Write 0x0A = 0000\_\textcolor{blue}{1010}b. Read 0x09 = 0000\_\textcolor{blue}{1001}b
  \end{center}
%  \vskip5mm
  \begin{center}
    \includegraphics[width=0.95\textwidth]
    {figures/sf2_starter_ujtag_ila_tdo_rise_0x0B_0x0A.png}\\
    (c) Write 0x0B = 0000\_\textcolor{blue}{1011}b. Read 0x0A = 0000\_\textcolor{blue}{1010}b
  \end{center}
%  \vskip5mm
  \begin{center}
    \includegraphics[width=0.95\textwidth]
    {figures/sf2_starter_ujtag_ila_tdo_rise_0x0C_0x0B.png}\\
    (d) Write 0x0C = 0000\_\textcolor{blue}{1100}b. Read 0x0B = 0000\_\textcolor{blue}{1011}b
  \end{center}
  \caption{JTAG logic analyzer traces for UTDO driven on UDRCK rising-edge.}
  \label{fig:sf2_starter_ujtag_ila_tdo_rise}
\end{figure}
% -----------------------------------------------------------------------------

\clearpage
% -----------------------------------------------------------------------------
% SmartFusion2 Starter Kit ILA traces for UTDO updated on DRCK falling-edge
% -----------------------------------------------------------------------------
%
\begin{figure}[p]
  \begin{center}
    \includegraphics[width=0.95\textwidth]
    {figures/sf2_starter_ujtag_ila_tdo_fall_0x09_0x10.png}\\
    (a) Write 0x09 = 0000\_\textcolor{blue}{1001}b. Read 0x10 = 000\textcolor{blue}{1\_000}\textcolor{red}{0}b
  \end{center}
%  \vskip5mm
  \begin{center}
    \includegraphics[width=0.95\textwidth]
    {figures/sf2_starter_ujtag_ila_tdo_fall_0x0A_0x13.png}\\
    (b) Write 0x0A = 0000\_\textcolor{blue}{1010}b. Read 0x13 = 000\textcolor{blue}{1\_001}\textcolor{red}{1}b
  \end{center}
%  \vskip5mm
  \begin{center}
    \includegraphics[width=0.95\textwidth]
    {figures/sf2_starter_ujtag_ila_tdo_fall_0x0B_0x14.png}\\
    (c) Write 0x0B = 0000\_\textcolor{blue}{1011}b. Read 0x14 = 000\textcolor{blue}{1\_010}\textcolor{red}{0}b
  \end{center}
%  \vskip5mm
  \begin{center}
    \includegraphics[width=0.95\textwidth]
    {figures/sf2_starter_ujtag_ila_tdo_fall_0x0C_0x17.png}\\
    (d) Write 0x0C = 0000\_\textcolor{blue}{1100}b. Read 0x17 = 000\textcolor{blue}{1\_011}\textcolor{red}{1}b
  \end{center}
  \caption{JTAG logic analyzer traces for UTDO driven on UDRCK falling-edge.}
  \label{fig:sf2_starter_ujtag_ila_tdo_fall}
\end{figure}
% -----------------------------------------------------------------------------

\clearpage
% -----------------------------------------------------------------------------
\subsection{UJTAG Testbench}
% -----------------------------------------------------------------------------

The simplest usage of the UJTAG component is a serial-to-parallel register
with write (host-to-FPGA) and read (FPGA-to-host) support. This sort of
interface is commonly found in SPI devices. For example, the Texas Instruments
LMX2615-SP clock synthesizer uses a 24-bit SPI transaction consisting of;
%
\begin{itemize}
\item 1-bit read(1)/write(0) indicator
\item 7-bit address (128 possible registers)
\item 16-bit data
\end{itemize}
%
Another example is the Texas Instruments LMK04832 clock synthesizer which
also uses a 24-bit SPI transaction, but with a slightly different bit
functionality;
%
\begin{itemize}
\item 1-bit read(1)/write(0) indicator
\item 15-bit address (32768 possible registers)
\item 8-bit data
\end{itemize}
%
The UJTAG component could be used to create this interface, but for the
purpose of this tutorial, just the serial-to-parallel register is
implemented, without interpretation of the read/write register bits.

The UJTAG testbench is located in the repository directory \verb+ip/ujtag+.
The UJTAG testbench is run as follows;
%
\begin{enumerate}
\item Start Questasim (2023.4 was used for this tutorial)
\item Change directory to the project, eg.,
\begin{verbatim}
vsim> cd {C:\github\microchip_jtag_tutorial\ip\ujtag}
\end{verbatim}
\item Source the simulation script, eg.,
\begin{verbatim}
vsim> source scripts/questasim.tcl
\end{verbatim}
The script ends with a list of the testbench procedures
\begin{verbatim}
# Testbench Procedures
# --------------------
#
#  ujtag_pa3_tb - run the UJTAG ProASIC3 testbench
#  ujtag_sf2_tb - run the UJTAG Smartfusion2 (and newer) testbench
#
\end{verbatim}
\item Run the ProASIC3 UJTAG testbench
\begin{verbatim}
vsim> ujtag_pa3_tb
\end{verbatim}
\item Run the SmartFusion2 UJTAG testbench
\begin{verbatim}
vsim> ujtag_sf2_tb
\end{verbatim}
\end{enumerate}
%
There are two UJTAG testbenches as the ProASIC3 (and earlier) devices used
a slightly different port naming convention than the SmartFusion2 (and newer)
devices. A UJTAG wrapper component is used to update the ProASIC3 port mapping
to match that of the SmartFusion2 port mapping, and the testbench instantiates
the wrapper. See the source code for additional details.

The UJTAG testbench implements the JTAG-to-Register component within the
testbench. The next section describes the JTAG-to-Register implemented as
a separate SystemVerilog module.

\clearpage
% =============================================================================
\section{JTAG-to-Register}
% =============================================================================
\label{sec:jtag_to_register}

% -----------------------------------------------------------------------------
\subsection{Architecture}
% -----------------------------------------------------------------------------

% -----------------------------------------------------------------------------
% JTAG-to-Register Block Diagram
% -----------------------------------------------------------------------------
%
\begin{figure}[t]
  \begin{center}
    \includegraphics[width=\textwidth]
    {figures/ujtag_to_register_diagram.pdf}
  \end{center}
  \caption{JTAG-to-Register block diagram.}
  \label{fig:jtag_to_register_diagram}
\end{figure}
% -----------------------------------------------------------------------------

The JTAG-to-Register component implements a single write (control) and
read (status) register. The register bit-width is controlled by a
SystemVerilog parameter, WIDTH. Figure~\ref{fig:jtag_to_register_diagram}
shows a block diagram of the architecture.
%
The JTAG-to-Register component uses a single-bit input select bit.
%
Figure~\ref{fig:jtag_to_register_diagram} shows the select bit will
assert for any valid user instruction (0x10 to 0x7F). User designs
could use the IR to select (and multiplex UTDO)  different blocks of
user logic.

Figure~\ref{fig:jtag_to_register_timing_ir} shows the instruction
shift-register timing. The TDO data shown in the figure is the
value read from the PolarFire SoC Discovery Kit.
%
Figure~\ref{fig:jtag_to_register_timing_ir} shows the data
shift-register timing. The figure shows TDI serializing 0x55 and
TDO serializing 0x11. These timing diagrams are reproduced by the
JTAG-to-Register testbench, and are discussed in the
Questasim waveform section.

% -----------------------------------------------------------------------------
% JTAG-to-Register Shift-IR
% -----------------------------------------------------------------------------
%
\begin{landscape}
\begin{figure}[p]
  \begin{center}
    \includegraphics[width=210mm]
    {figures/ujtag_to_register_timing_ir.pdf}
  \end{center}
  \caption{JTAG-to-Register JTAG instruction shift-register timing.}
  \label{fig:jtag_to_register_timing_ir}
\end{figure}
\end{landscape}
% -----------------------------------------------------------------------------

% -----------------------------------------------------------------------------
% JTAG-to-Register Shift-DR
% -----------------------------------------------------------------------------
%
\begin{landscape}
\begin{figure}[p]
  \begin{center}
    \includegraphics[width=210mm]
    {figures/ujtag_to_register_timing_dr.pdf}
  \end{center}
  \caption{JTAG-to-Register JTAG data shift-register timing.}
  \label{fig:jtag_to_register_timing_dr}
\end{figure}
\end{landscape}
% -----------------------------------------------------------------------------

\clearpage
% -----------------------------------------------------------------------------
\subsection{Testbench}
% -----------------------------------------------------------------------------

The JTAG-to-Register component is general-purpose, so is located in the
repository intellectual property (IP) directory \verb+ip/jtag_to_register+.
The JTAG-to-Register testbench is run as follows;
%
\begin{enumerate}
\item Start Questasim (2023.4 was used for this tutorial)
\item Change directory to the project, eg.,
\begin{verbatim}
vsim> cd {C:\github\microchip_jtag_tutorial\ip\jtag_to_register}
\end{verbatim}
\item Source the simulation script, eg.,
\begin{verbatim}
vsim> source scripts/questasim.tcl
\end{verbatim}
The script ends with a list of the testbench procedures
\begin{verbatim}
# Testbench Procedures
# --------------------
#
#  jtag_to_register_tb - run the JTAG-to-Register testbench
#
\end{verbatim}
\item Run the testbench
\begin{verbatim}
vsim> jtag_to_register_tb
\end{verbatim}
\end{enumerate}
%
The testbench uses the SmartFusion2 UJTAG component.

\clearpage
% -----------------------------------------------------------------------------
\subsection{Waveforms}
% -----------------------------------------------------------------------------

% -----------------------------------------------------------------------------
% JTAG-to-Register Questasim Waveforms
% -----------------------------------------------------------------------------
%
\begin{figure}[t]
  \begin{center}
    \includegraphics[width=\textwidth]
    {figures/jtag_to_register_questasim_waveforms.png}
  \end{center}
  \caption{JTAG-to-Register testbench Questasim waveforms.}
  \label{fig:jtag_to_register_questasim_waveforms}
\end{figure}
% -----------------------------------------------------------------------------

Figure~\ref{fig:jtag_to_register_questasim_waveforms} shows the Questasim
waveform view captured from the JTAG-to-Register testbench, with the
JTAG-to-Register component configured for 8-bits control/status width.
%
The testbench interfaces to the JTAG signals and implements the following
sequence;
%
\begin{itemize}
\item TAP reset
\item Load the 8-bit instruction register with 0x20
\item Load the 8-bit data register with 0x11 = 0001\_0001b
\item Load the 8-bit data register with 0x55 = 0101\_0101b
\item Load the 8-bit data register with 0x99 = 1001\_1001b
\item TAP reset
\end{itemize}
%
Figure~\ref{fig:jtag_to_register_questasim_waveforms} shows the data
register load with 0x55. Items of interest in the
waveform are;
%
\begin{itemize}
\item TDI contains the LSB-to-MSB serial data 0x55 = 0101\_0101b
\item TDO contains the LSB-to-MSB serial data 0x11 = 0001\_0001b
\item UTDO updates on the rising-edge of TCK
\item TDO updates on the falling-edge of TCK
\item status is captureed during the CAPTURE TAP state
\item control is updated during the UPDATE TAP state
\end{itemize}
%

\clearpage
% -----------------------------------------------------------------------------
\subsection{Software Interfacing with STAPL}
% -----------------------------------------------------------------------------

% -----------------------------------------------------------------------------
% Polarfire SoC Discovery and the Digilent Arty board
% -----------------------------------------------------------------------------
%
\begin{figure}[t]
  \begin{center}
    \includegraphics[width=\textwidth]
    {figures/pfs_disco_and_arty.png}\\
  \end{center}
  \caption{PolarFire SoC Discovery Kit and the Digilent Arty Board.}
  \label{fig:pfs_disco_and_arty}
\end{figure}
% -----------------------------------------------------------------------------

STAPL (Standard Test and Programming Language) is an Altera-developed standard
for JTAG interfacing, standardized in JEDEC standard
JESD-71~\cite{JEDEC_JESD71_1999}.
%
Two STAPL files are located in the repository:
%
\begin{itemize}
\item \verb+stp/read_idcode.stp+

Read and print a Microchip JTAG device IDCODE.

This script resets the TAP, loads the instruction register with the 8-bit
read IDCODE instruction, and then reads a single 32-bit IDCODE.

\item \verb+stp/jtag_to_register.stp+

Incrementing LED count.

This script resets the TAP, loads the instruction register with the 8-bit
user instruction 0x20, and then loops from 1 to 15, loading the data register
with the loop index, and printing the value read (which matches the previously
loaded loop index).

\end{itemize}
%
The first version of the JTAG-to-Register STAPL was based on the example in
AC227 Appendix A~\cite{Microchip_AC227_2015}. The JEDEC STAPL specification
was then consulted to understand the syntax for: loops, integer-to-binary
conversion, binary-to-integer conversion, and printing messages.

Figure~\ref{fig:pfs_disco_and_arty} shows the hardware setup used to
capture logic analyzer traces. The PolarFire SoC Discovery Kit Raspberry
Pi connector drove UJTAG signals to a PMod connector on the
Digilent Arty board, and the Xilinx Vivado
\href{https://www.xilinx.com/products/intellectual-property/ila.html}
{Integrated Logic Analyzer (ILA)} was used to capture traces.
%
FlashPro Express was used to execute the JTAG-to-Register STAPL file.
Figure~\ref{fig:jtag_to_register_stapl} shows logic analyzer traces for
TDI serializing 0x02, 0x03, and 0x04, and TDO serializing 0x01, 0x02, and
0x03.

\clearpage
% -----------------------------------------------------------------------------
\subsection{Software Interfacing with FTDI D2XX}
% -----------------------------------------------------------------------------

A JTAG-to-Register application was written in C++, compiled with Microsoft
Visual Studio 2022 (Community Edition), and linked with the FTDI D2XX DLL
(the application is part of the FTDI D2XX tutorial).
%
The application used the FTDI Multi-Protocol Synchronous Serial Engine
(MPSSE) mode to implement JTAG transactions.
%
Figure~\ref{fig:jtag_to_register_d2xx} shows logic analyzer traces for
TDI serializing 0x02, 0x03, and 0x04, and TDO serializing 0x01, 0x02, and
0x03. The logic analyzer traces are very similar to those captured from the
FlashPro Express STAPL transactions in
Figure~\ref{fig:jtag_to_register_stapl}, with one minor difference in
that the STAPL captures show additional TCK pulses while in Run-Test-Idle.

% -----------------------------------------------------------------------------
% JTAG-to-Register on the Polarfire SoC
% -----------------------------------------------------------------------------
%
\begin{figure}[p]
  \begin{center}
    \includegraphics[width=0.95\textwidth]
    {figures/jtag_to_register_stapl_0x02_0x01.png}\\
    (a) Write 0x02 = 0000\_0010b. Read 0x01 = 0000\_0001b
  \end{center}
  \vskip5mm
  \begin{center}
    \includegraphics[width=0.95\textwidth]
    {figures/jtag_to_register_stapl_0x03_0x02.png}\\
    (b) Write 0x03 = 0000\_0011b. Read 0x02 = 0000\_0010b
  \end{center}
  \vskip5mm
  \begin{center}
    \includegraphics[width=0.95\textwidth]
    {figures/jtag_to_register_stapl_0x04_0x03.png}\\
    (c) Write 0x04 = 0000\_0100b. Read 0x03 = 0000\_0011b
  \end{center}
  \caption{JTAG-to-Register logic analyzer traces for STAPL interfacing.}
  \label{fig:jtag_to_register_stapl}
\end{figure}
% -----------------------------------------------------------------------------

% -----------------------------------------------------------------------------
% JTAG-to-Register on the Polarfire SoC
% -----------------------------------------------------------------------------
%
\begin{figure}[p]
  \begin{center}
    \includegraphics[width=0.95\textwidth]
    {figures/jtag_to_register_d2xx_0x02_0x01.png}\\
    (a) Write 0x02 = 0000\_0010b. Read 0x01 = 0000\_0001b
  \end{center}
  \vskip5mm
  \begin{center}
    \includegraphics[width=0.95\textwidth]
    {figures/jtag_to_register_d2xx_0x03_0x02.png}\\
    (b) Write 0x03 = 0000\_0011b. Read 0x02 = 0000\_0010b
  \end{center}
  \vskip5mm
  \begin{center}
    \includegraphics[width=0.95\textwidth]
    {figures/jtag_to_register_d2xx_0x04_0x03.png}\\
    (c) Write 0x04 = 0000\_0100b. Read 0x03 = 0000\_0011b
  \end{center}
  \caption{JTAG-to-Register logic analyzer traces for FTDI D2XX interfacing.}
  \label{fig:jtag_to_register_d2xx}
\end{figure}
% -----------------------------------------------------------------------------



% -----------------------------------------------------------------------------
\subsection{Hardware Implementations}
% -----------------------------------------------------------------------------

Table~\ref{tab:jtag_to_register_hardware} shows the JTAG-to-Register
design was implemented on the following hardware:
%
\begin{itemize}
\item Microchip Polarfire SoC Discovery Kit
(\href{https://www.microchip.com/en-us/development-tool/mpfs-disco-kit}
{MPFS-DISCO-KIT})

\item Microchip ProASIC3 Starter Kit
(\href{https://www.microchip.com/en-us/development-tool/a3pe-starter-kit-2}
{A3PE-STARTER-KIT-2})

\item Microchip SmartFusion2 Security Evaluation Kit
(\href{https://www.microchip.com/en-us/development-tool/m2s090ts-eval-kit}
{M2S090TS-EVAL-KIT})

\item Emcraft SmartFusion2 Starter Kit
(\href{https://emcraft.com/products/153}{SF2-STARTER-KIT})

The tutorial used the ES version of this kit: SF2-STARTER-KIT-ES

\item Avnet SmartFusion2 Kickstart Kit (no longer available)

\end{itemize}
%
The UJTAG component is supported by other Microchip FPGAs, eg.,
\href{https://www.microchip.com/en-us/development-tool/m2gl-eval-kit}{IGLOO2}.

% -----------------------------------------------------------------------------
% Hardware implementations
% -----------------------------------------------------------------------------
%
\begin{table}
\caption{JTAG-to-Register hardware implementations.}
\label{tab:jtag_to_register_hardware}
\begin{center}
\begin{tabular}{|l|l|l|}
\hline
Directory & Board & Control\\
\hline\hline
&&\\
\texttt{designs/pfs\_disco}     & PolarFire SoC Discovery & 4 LEDs\\
\texttt{designs/pa3\_starter}   & ProASIC3 Starter        & TBD\\
\texttt{designs/sf2\_security}  & SmartFusion2 Security   & 4 LEDs\\
\texttt{designs/sf2\_starter}   & SmartFusion2 Starter    & 1 LED\\
\texttt{designs/sf2\_kickstart} & SmartFusion2 Kickstart  & 4 LEDs\\
&&\\
\hline
\end{tabular}
\end{center}
\end{table}
% -----------------------------------------------------------------------------

\clearpage
% =============================================================================
\section{Epilogue}
% =============================================================================
\label{sec:epilogue}

This tutorial (and associated design files) demonstrates the use of the
Microchip UJTAG component: simulation, synthesis, and hardware
implementation.
%
The UJTAG component and the JTAG-to-Register example form the basis
for more complex JTAG interface components:
%
\begin{itemize}
\item \textbf{JTAG-to-AXI-Stream Bridge}

The JTAG-to-AXI-Stream bridge interfaces JTAG serial data streams into
byte streams compiliant with the ARM
\href{https://developer.arm.com/documentation/ihi0051/latest}
{AMBA AXI-Stream Protocol Specification}.
%
The JTAG TDO bit-stream is converted to an AXI-Stream Transmitter
byte-stream, while the JTAG TDI bit-stream is used to receive data
from an AXI-Stream Receiver byte-stream.

\item \textbf{JTAG-to-AXI Bridge}

The JTAG-to-AXI bridge interfaces JTAG serial data streams into transactions
compliant with the ARM
\href{https://developer.arm.com/documentation/ihi0022/latest}
{AMBA AXI Protocol Specification}.
%
The JTAG-to-AXI bridge combines the JTAG-to-AXI-Stream Bridge with an
AXI-Stream-to-AXI Manager.

\end{itemize}
%
These components can be found on the same github as this tutorial.


\clearpage
\appendix
% -----------------------------------------------------------------------------
\section{Revision History}
% -----------------------------------------------------------------------------
%
\begin{table}[h]
%\caption{Revision History}
\begin{center}
\begin{tabular}{|c|p{100mm}|}
\hline
Date & Description\\
\hline\hline
&\\
01/15/2025  & Created the document.\\
01/19/2025  & Completed the UJTAG and JTAG-to-Register sections.\\
&\\
\hline
\end{tabular}
\end{center}
\end{table}

%\clearpage
%------------------------------------------------------------------------------
% Do the bibliography
%------------------------------------------------------------------------------
%
% Note, you can't have spaces in the list of bibliography files
%
\bibliography{sections/refs}
\bibliographystyle{plain}

%------------------------------------------------------------------------------
\end{document}
